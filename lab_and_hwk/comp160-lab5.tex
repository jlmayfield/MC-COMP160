\documentclass[nobib]{tufte-handout}
\usepackage{amsmath,amssymb,amsthm}
\usepackage{listings}
\usepackage{enumitem}

\hypersetup{
  colorlinks = true,
}

\lstset{
  frame=tb,
  language=Lisp,
  aboveskip=3mm,
  belowskip=3mm,
  showstringspaces=false,
  columns=flexible,
  basicstyle={\small\ttfamily},
  numbers=left,
  stepnumber=1,
  firstnumber=1,
  numbersep=5pt,
  numberfirstline=true,
  numberstyle=\tiny\color{black},
  keywordstyle=\color{blue},
  commentstyle=\color{gray},
  stringstyle=\color{green},
  breaklines=true,
  breakatwhitespace=true,
  tabsize=3
}

\title{Comp160 \\ Lab 5 }
\author{}
\date{ Spring 2019 }



\begin{document}
\maketitle

\begin{abstract}
For this lab you'll be working with defining and using structs in Racket programs.  If you have not already, you might need to read Sections 5.1 through 5.5 and review your notes from class.
\end{abstract}

\section*{Lab 5}

Do the following:
\begin{itemize}
  \item \href{https://htdp.org/2018-01-06/Book/part_one.html#%28counter._%28exercise._struct2%29%29}{Exercise 64} Be certain to provide a signature, statement of purpose, and tests for the function.  Write a header for practice as well.
  \item \href{https://htdp.org/2018-01-06/Book/part_one.html#%28counter._%28exercise._struct3%29%29}{Exercise 65} Just do Person, CD, and sweater
  \item \href{https://htdp.org/2018-01-06/Book/part_one.html#%28counter._%28exercise._struct3b%29%29}{Exercise 66} Again, just for Person, CD, and sweater.
  \item \href{https://htdp.org/2018-01-06/Book/part_one.html#%28counter._%28exercise._struct4%29%29}{Exercise 67}
  \item \href{https://htdp.org/2018-01-06/Book/part_one.html#%28counter._%28exercise._struct5%29%29}{Exercise 68}
  \item \href{https://htdp.org/2018-01-06/Book/part_one.html#%28counter._%28exercise._struct3a%29%29}{Exercise 69} Again, just for Person, CD, and sweater.
\end{itemize}


\end{document}
