\documentclass[nobib]{tufte-handout}
\usepackage{amsmath,amssymb,amsthm}
\usepackage{listings}
\usepackage{enumitem}

\hypersetup{
  colorlinks = true,
}

\lstset{
  frame=tb,
  language=Lisp,
  aboveskip=3mm,
  belowskip=3mm,
  showstringspaces=false,
  columns=flexible,
  basicstyle={\small\ttfamily},
  numbers=left,
  stepnumber=1,
  firstnumber=1,
  numbersep=5pt,
  numberfirstline=true,
  numberstyle=\tiny\color{black},
  keywordstyle=\color{blue},
  commentstyle=\color{gray},
  stringstyle=\color{green},
  breaklines=true,
  breakatwhitespace=true,
  tabsize=3
}

\title{Comp160 \\ Lab 7 }
\author{}
\date{ Spring 2019 }



\begin{document}
\maketitle

\begin{abstract}
In this lab you'll continue working on the UFO game discussed at the start of chapter 6 and in class. This program design uses an itemization of structures as the main worldstate data type.  Such a design lets you continue to hone your skills using our core templates for itemization data, structure data, and composed data types.
\end{abstract}

\section*{Lab 7}

In class we discussed the basic design of this program and effectively covered exercises 94--96. Your goal for lab is to get through exercises 97 and 98. If you have time work on exercises 99 and 100. We will finish all of these in class by the end of the week.  Regardless of where you are at by the end of lab, get your program into stable state (it should run without error but may contain failing tests) and submit your work before you leave the lab. You're required to make good, effective progress on the function design tasks.  You are not required to complete them all. What I should not see is completed function definitions without tests. This means you're not working the design process in the manner directed by the text.  Before you begin work on the excises you must: collect all the necessary data definitions and templates, define appropriate program constants, and setup a basic main function for the program. In short, you need to make your own starter this time such that you can begin the function design tasks from the same starting place as last time.

\end{document}
