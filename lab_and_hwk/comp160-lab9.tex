\documentclass[nobib]{tufte-handout}
\usepackage{amsmath,amssymb,amsthm}
\usepackage{listings}
\usepackage{enumitem}

\hypersetup{
  colorlinks = true,
}

\lstset{
  frame=tb,
  language=Lisp,
  aboveskip=3mm,
  belowskip=3mm,
  showstringspaces=false,
  columns=flexible,
  basicstyle={\small\ttfamily},
  numbers=left,
  stepnumber=1,
  firstnumber=1,
  numbersep=5pt,
  numberfirstline=true,
  numberstyle=\tiny\color{black},
  keywordstyle=\color{blue},
  commentstyle=\color{gray},
  stringstyle=\color{green},
  breaklines=true,
  breakatwhitespace=true,
  tabsize=3
}

\title{Comp160 \\ Lab 9 }
\author{}
\date{ Spring 2019 }



\begin{document}
\maketitle

\begin{abstract}
For this lab you'll look at three list-based functions each of which represent common list processing idioms.
\end{abstract}

\section*{Lab 9}

The overall problem we want to solve is counting the number of strings in a list of strings that are longer then some given length. To do this we'll break the work
into three sub-tasks: determine the length of every string in the list, compute a list containing on those strings that are long enough, then counting the length
of a string.  These subtasks are examples of the \textit{map}, \textit{filter}, and \textit{reduce} idioms respectively. A map is function that computes something about each element in a list and produces a list containing those results. In this case, we're computing the lengths of every string in a list of strings. A filter selects out all the elements in a list that meet a particular criteria. In this case we're filtering out all the numbers that are greater than or equal to a given value. Finally, a reduce effectively reduces a list down to some other piece of information.  In this case we're reducing the list down to it's length.  A great number of problems can be rethought in terms of some combination of \textit{map}, \textit{filter}, and \textit{reduce}. After this lab we'll talk more about the patterns inherent in each idiom.

Complete the following tasks in the order in which they are presented.
\begin{enumerate}
  \item Write the data type definition for a list of strings.
  \item Write out the template for functions acting on a list of strings.
  \item Write out the data type definition for a list of PositiveNumbers.
  \item Write out the template for functions acting on a list of PositiveNumbers.
  \item Produce the signature, purpose, and header for each of these functions:
  \begin{enumerate}
    \item \textit{count-longer-than} - A function that takes a list of strings and a length and counts the number of strings that are at least as long as the given length.
    \item \textit{length-map} - A function that takes a list of strings and computes
      the list containing the length of every string in the given list of strings.
    \item \textit{at-least-filter} - A function that takes a list of PositiveNumbers and a PositiveNumbers and computes the list of all the numbers in the given list that are greater than or equal to the given PositiveNumber.
    \item \textit{list-length-reduce} - A function that takes a list of PositiveNumbers and computes the length of the given list.
  \end{enumerate}
  \item Write tests for all of the functions listed above.
  \item Complete the definition for the functions listed above.  The function \textit{count-longer-than} should first use the map, then the filter, then the reduce functions to solve the problem.  It does not use the list template.
\end{enumerate}

\end{document}
