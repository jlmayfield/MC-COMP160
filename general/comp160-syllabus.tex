\documentclass[10pt]{article}
\usepackage{amsmath}
\usepackage{setspace}
\usepackage{hyperref}
\usepackage{booktabs}

\setlength{\textheight}{9in} \setlength{\topmargin}{-.5in}
\setlength{\textwidth}{6.5in} \setlength{\oddsidemargin}{0in}
\setlength{\evensidemargin}{0in}

\title{Syllabus \\ COMP 160 \\ Introduction to Computer Science}
\author{  }
\date{Spring 2018}

\begin{document}
\maketitle

\section{Logistics}
\begin{itemize}
\item \textbf{Where: }
\begin{itemize}
\item Class: Center for Science and Business (CSB), Room 309
\item Lab: Center for Science and Business (CSB), Room 309
\end{itemize}
\item \textbf{When: }
\begin{itemize}
  \item Class: MWF 8--8:50am
  \item Lab 1: M 2--4:50pm
  \item Lab 2: T 2--4:50pm
\end{itemize}
\item \textbf{Instructor: } James \textit{Logan} Mayfield
\begin{itemize}
\item \textit{Office: } Center for Science and Business (CSB), Room 344
\item \textit{Phone: } 309-457-2200 % chktex 8
\item \textit{Website: } \url{http://jlmayfield.github.io/}
\item \textit{Email: } lmayfield \textit{at} monmouthcollege \textit{dot} edu
\item \textit{Office Hours: }  By Appointment.
\end{itemize}
\item \textbf{Website: } \url{http://jlmayfield.github.io/teaching/COMP160/}
\item \textbf{Credits: } 1 course credit
\end{itemize}
\emph{Note: This Syllabus is subject to change based on specific class needs. Significant deviations from the syllabus will be discussed in class.}

\section{Textbook}

\noindent
Felleisen, Matthias, Et al\@. \textit{How to Design Programs, Second Edition}. MIT Press. 2014. \url{http://www.ccs.neu.edu/home/matthias/HtDP2e/} % chktex 8

\section{Description and Content}

This is the first of a three course sequence required for the Computer Science major and minor. In this course we introduce the basic building blocks of computational processes and how to express them as computer programs in a functional programming paradigm. The emphasis is on a \textit{reasoned program design methodology} and not on details of the programming language. Towards this end, students will work with a subset of the programming language Racket that is designed for beginning programmers. This is very much a learn by doing course. Nearly every week of the semester, students will spend several hours in the lab working on programming problems under the guidance and supervision of the instruction and a lab tutor. In addition to labs, students will spend time outside of class working exercises and project. By the end of the semester students will have covered the first twelve chapters of the text.

\section{Expectations and Policies}

Students are expected to carry themselves in a mature and professional manner in this course. Towards this end, there are a few classroom policies by which every student is expected to abide.
\begin{itemize}

\item \textit{Late Assignments: } In general, late assignments will \textit{not} be accepted.  Students who feel they have a justified reason for submitting an assignment late may set up an appointment to meet with the instructor and plead their case.  Students are more likely to get extensions on assignments when they are asked for in advance rather than the day the assignment is due.

\item \textit{Attendance: } \textbf{Repeated absences and late arrivals to class will quickly reduce a student's participation grade to zero.}  The occasional late arrival or missed class is one thing, but being habitually late and regularly missing classes is disruptive and not fair to your classmates.

\item \textit{Participation: }  Cellphone and computer usage in class for non-class related activities is strongly discouraged.  All devices should be set to silent when in class.  If a student's usage of technology becomes a distraction to their classmates or the instructor, then that student's participation grade will suffer.  If the instructor or a classmate has to inform a student that they're being a distraction, then their use of technology has already gone too far.  When in doubt, err on the side of caution.

\end{itemize}


\subsection{Collaboration}

In general, students are encouraged to make use of the resources available to them.  This means it is OK to seek help from a friend, the tutor, the instructor, the internet, etc.  However, \textit{copying of answers and any act worthy of the label of ``cheating'' or ``plagiarism'' is never permissible!. Students should always be able to reproduce an answer on their own, and if they cannot then they likely \textbf{do not really known the material.}} All of the Monmouth College rules on academic dishonesty apply.  A student found in violation of the rules should be prepared to face the consequences of their actions. If a student needs help understanding the rules, then please seek out the instructor before doing something that might violate academic honesty policies.

\section{Grades}

This courses uses a standard grading scale.  Assignments and final grades will not be curved except in rare cases when its deemed necessary by the instructor.  Percentage grades translate to letter grades as follows:

\begin{center}
\begin{small}
\begin{tabular}{lcl}
\underline{Score} & & \underline{Grade} \\ 
94--100 & & A \\
90--93 & & A- \\
88--89 & & B+ \\
82--87 & & B \\
80--81 & & B- \\
78--79 & & C+ \\
72--77 & & C \\
70--71 & & C- \\
68--69 & & D+ \\
62--67 & & D \\
60--61 & & D- \\
0--59 & & F
\end{tabular}
\end{small}
\end{center}


You are always welcome to challenge a grade that you feel is unfair or calculated incorrectly.  Mistakes made in your favor will never be corrected to lower your grade.  Mistakes made not in your favor will be corrected.  \textit{Basically, after the initial grading, your score can only go up as the result of a challenge.}

\subsection{Workload}
% number of/details on midterms, finals, project, homeworks, quizes, etc

The course workload is as follows:
\begin{center}
  \begin{tabular}{ll}
    \underline{Category} & \underline{Number of Assignments} \\     Labs & 10 \\
    Homework & 7 \\
    Projects & 2 \\
    Exams & 7
  \end{tabular}
\end{center}

There will be no dedicated midterm or final exam. There are just exams.  Exams will generally focus on material covered since the previous exam.

\subsection{Grade Weights}

Your final grade is based on a weighted average of particular assignment categories.  You should be able to estimate your current grade based on your scores and these weights.  You may always visit the instructor \textit{outside of class time} to discuss your current standing.

\begin{center}
  \begin{tabular}{ll}
  \underline{Category} & \underline{Weight} \\
    Exams & 42\% \\ %6 each
    Projects & 25\% \\ %12.5 each
    Homework & 14\% \\ %1.5 each
    Labs & 14\% \\ %1.25 each
    Participation & 5\%
  \end{tabular}
\end{center}


\subsection{Course Engagement Expectations}

The weekly workload for this course will vary by student but on average should be about 13 hours per week.  The follow tables provides a rough estimate of the distribution of this time over different course components for a 16 week semester.
\begin{center}
\begin{tabular}{ll}
\underline{Assignment Type} & \underline{Time/week} \\
Lectures+Labs       & 6 hours/week \\
Homework          & 2 hours/week \\
Exam Study Time    & 1 hours/week \\
Projects          & 3 hours/week \\
Reading+Unstructured Study &  1 hours/week \\
\bottomrule
 & 13 hours/week
\end{tabular}
\end{center}


\subsection{Calendar}

\textit{This calendar is subject to change based on the circumstances of the course.}

\begin{center}
\begin{tabular}{llll}
\underline{Week} & \underline{Dates} & \underline{Assignments Due} & \underline{Chapter(s)}\\
1 & 1/15 --- 1/19 & Lab 1. &  Prologue. 2.5\\
2 & 1/22 --- 1/26 & Lab 2. &  1 \\
3 & 1/29 --- 2/2 & Hwk 1. Lab 3. Exam 1 (F).  &  2 \\
4 & 2/5 --- 2/9 & Lab 4. Hwk 2. & 4.1--4.5 \\
5 & 2/12 --- 2/16 & Exam 2 (M). Lab 5. Hwk 3. & 5.1--5.5, 3.1\\
6 & 2/19 --- 2/23 & Lab 6. Exam 3 (W).  & 3 \\
7 & 2/26 --- 3/2 & Lab 7. Hwk 4. Exam 4 (F). & 4.6--7  \\
8 & 3/5 --- 3/9 & SPRING BREAK & SPRING BREAK \\
9 & 3/12 --- 3/16 & & 5 \\
10 & 3/19 --- 3/23 & Lab 8. Project 1. Hwk 5. & 6,7 \\
11 & 3/26 --- 3/29 & Lab 9. Exam 5 (W). EASTER (F)  & 8 \\
12 & 4/3 --- 4/6 & EASTER (M). & 8,9 \\
13 & 4/9 --- 4/13 & Lab 10.  Hwk 6. & 10  \\
14 & 4/16 --- 4/20 & Exam 6 (M).  & 11 \\
15 & 4/23 --- 4/27 & SCHOLAR'S DAY (Tu). Hwk 7.  & 11,12 \\
16 & 4/30 --- 5/2 & Project 2. READING DAY (Th) & 12 \\
Final's Week & 5/7 (6:30pm --- 9:30pm) & Exam 7.  &  \\
\end{tabular}
\end{center}

\end{document}
